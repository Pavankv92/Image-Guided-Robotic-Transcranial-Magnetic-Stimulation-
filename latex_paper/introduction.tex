% !TEX root = paper.tex
\section{Introduction [LK, MH, SE]}

% \addcontentsline{toc}{section}{Introduction} % Adds this section to the table of contents

Transcranial magnetic stimulation (TMS) is a medical procedure which uses magnetic fields to stimulate a specific cortex region. During this procedure, it is critical to keep the coil fixed to the relevant section of the head.
This usually requires the patient to avoid head movements for up to 30 minutes. However, it was found, that the electric field in the target region is reduced by 32.0\% at the end of the treatment \cite{TMS_Schlaefer}.
Using a robot to place the coil and an image guiding system to maneuver the robot to keep it in the correct position can improve the effectiveness and make it more comfortable for the patient.
We implement such a motion compensating system and evaluate its performance in a physical test setup (fig. \ref{headtrackingsetup}).

\begin{figure}[ht]
	\centering
 	\includegraphics[width=\linewidth]{headtracking_image}
 	\caption{Laboratory test setup}
 	\label{headtrackingsetup}
 \end{figure}
The used methods to create the described motion compensation system are explained in section 2. In the third section we present and discuss our results. A conclusion is drawn in the fourth section.

% \lipsum[1-3] % Dummy text
